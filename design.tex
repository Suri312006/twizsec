% CHANGE THESE DEFINITIONS
\newcommand{\NAME}{Surendra Jammishetti}
\newcommand{\ASSIGNMENT}{TwizSec Library Crate}
\newcommand{\CLASS}{Twizzler}


%%%%%%%%%%%%%%%%%%%%%%%%%%%%%%%%%%%%%%%%%%%%%

\documentclass{article}
\usepackage{graphicx} % Required for inserting images
\usepackage{multirow}
\usepackage{hyperref}
\usepackage{lastpage}
\usepackage{fancyhdr}
\usepackage{geometry}
\geometry{margin=1in}
\usepackage{underscore}
\usepackage{subcaption}
\usepackage{fancyvrb}
\usepackage{listings}
\lstset{
basicstyle=\small\ttfamily,
columns=flexible,
breaklines=true
}


\title{\ASSIGNMENT}
\author{\NAME}
\date{\CLASS}

\begin{document}
\pagestyle{fancy}
\fancyfoot{}
\fancyhead{}
\fancyfoot[L]{\ASSIGNMENT\ -- \CLASS\ -- \NAME}
\fancyfoot[R]{\thepage}

\maketitle


\section{Early Goals}

The TwizSec crate aims to provide an external library for the Twizzler kernel that has
the following goals (summarized from me and Daniels meeting on 11/26/24).
\begin{enumerate}
	\item storing and receiving capabilities
	\item signing and verifying capabilities
	\item programming the mmu / io to reflect security policy data
\end{enumerate}

\section{Planning}


\subsection{Needs}
The plan is to work on the second item first, as its the path of least resistance.
Ideally expose two functions.
\begin{enumerate}
	\item Takens in capability and signature, returns if they are correct or not
	\item Given a capability, construct a signature
\end{enumerate}

\subsection{Deps}
The kernel has crypto libraries already integrated, use those to build these features
currently this is all we got

p256 : https://crates.io/crates/p256

sha2 : https://crates.io/crates/sha2

Which, atleast right now, should have everything we need.

\subsection{Capabilities}

Currently we dont have a capability struct, so Im going to use
what was in the security paper as an example.

Additionally I'm considering making the two functions impl'd onto the struct,
so that way they can be called on any capability struct, as I think it would
be nice and ergonomic but not sure what others would think.

This is the spec inside the paper

\begin{verbatim}
CAP := {
    target, accessor : ObjectId,
    permissions, flags : BitField,
    gates: Gates,
    revocation : Revoc,
    siglen: Length,
    sig: u8[],
}
\end{verbatim}


\section{Explaining the Implementation}

\subsection{Cap struct}
The Cap struct encompasses the functionality of the Capability, having
everything but the gates and revocation fields present in the paper, only because
I wasnt planning to finish that this time around, maybe next time. I'll list some
key decision details below.
\begin{itemize}
	\item UnsignedCapability =$>$ Capability VS Capability::new()


	      My first iteration had a UnsignedCap struct that would only have two methods,
	      new(...) -$>$ Self , which would initialize it, and a sign(self) -$>$ Cap, which would consume
	      the unsignedcap and return a signed capability. The second implementation would
	      only have the capability struct, which gets signed on initialization. While
	      I liked how "satisfying" the sign function was in the first implementation,
	      the conciseness of the second implementation is cleaner
	      since its one struct, one function, one capability, so I went ahead with that.

	\item The Serialize Function

	      Since its a no_std environment, the most pressing concern is that there is no
	      std library, nor any memory allocator. Which means that when im trying to
	      "serialize" the contents of the capability struct to hash (where all of
	      the hashing functions take in a [u8]). The easiest solution, in my mind atleast,
	      was to just count how many bytes the array is and just create a new array of that size,
	      copy the bytes of each field over one by one, and boom, serialized capability struct as
	      fast as possible. Just note that once the revoc and gates fields are added, this
	      hashing function will need to be changed to account for them.

	\item Verification function

	      I tried to follow the spec in the security paper as much as I could, the basic steps I take are
	      \begin{itemize}
		      \item verify_sig(\&self, verifying_key) -> Result<(), CapError>; // function signature
		      \item parse hashing_algo and signing scheme from capability flags
		      \item serialize self and store in buffer
		      \item use the hashing_algo on the buffer to get the capabilities hash
		      \item ensure that signing_scheme matches the signing_scheme on verifying_key
		      \item verify the signature according to the signing scheme using the verifying key
	      \end{itemize}

	      originally this function took in the private key of the target object to generate the
	      verification key but now, due to Daniel's advice, It takes in a verification
	      key, meaning we only have to generate the verification keys once and can reuse them
	      to verify capabilities
	\item Errors

	      Currently the errors are generalized to the entire library and are very
	      simplistic, which is good but might need to consider expanding them once
	      theres more parts of the sec model implemented.

	\item testing
	      Theres only one pretty simple test that creates a capability for one object to access another,
	      and then generates a verification key and tests if the capability can be verified. It works!

	\item Benchmarking
	      While yes its fairly early to have benchmarks, fast code => happiness so I imported
	      a reliable benchmarking crate as a dev-dependency and tested
	      some core capability ops (can be found in /benches/capability_ops.rs).
	      Good news, its pretty good!
\end{itemize}


\subsection{Literally anything else}
check out some of the code comments or message me and id gladly answer.

\end{document}
