% CHANGE THESE DEFINITIONS
\newcommand{\NAME}{Surendra Jammishetti}
\newcommand{\ASSIGNMENT}{TwizSec Library Crate}
\newcommand{\CLASS}{Twizzler}


%%%%%%%%%%%%%%%%%%%%%%%%%%%%%%%%%%%%%%%%%%%%%

\documentclass{article}
\usepackage{graphicx} % Required for inserting images
\usepackage{multirow}
\usepackage{hyperref}
\usepackage{lastpage}
\usepackage{fancyhdr}
\usepackage{geometry}
\geometry{margin=1in}
\usepackage{underscore}
\usepackage{subcaption}
\usepackage{fancyvrb}
\usepackage{listings}
\lstset{
basicstyle=\small\ttfamily,
columns=flexible,
breaklines=true
}


\title{\ASSIGNMENT}
\author{\NAME}
\date{\CLASS}

\begin{document}
\pagestyle{fancy}
\fancyfoot{}
\fancyhead{}
\fancyfoot[L]{\ASSIGNMENT\ -- \CLASS\ -- \NAME}
\fancyfoot[R]{\thepage}

\maketitle


\section{Early Goals}

The TwizSec crate aims to provide an external library for the Twizzler kernel that has
the following goals (summarized from me and Daniels meeting on 11/26/24).
\begin{enumerate}
    \item storing and receiving capabilities
    \item signing and verifying capabilities
    \item programming the mmu / io to reflect security policy data
\end{enumerate}

\section{Implementation}


\subsection{Needs}
The plan is to work on the second item first, as its the path of least resistance.
Ideally expose two functions.
\begin{enumerate}
    \item Takens in capability and signature, returns if they are correct or not
    \item Given a capability, construct a signature
\end{enumerate}

\subsection{Deps}
The kernel has crypto libraries already integrated, use those to build these features
currently this is all we got

p256 : https://crates.io/crates/p256

sha2 : https://crates.io/crates/sha2

Which, atleast right now, should have everything we need.

\subsection{Capabilites}

Currently we dont have a capability struct, so Im going to use
what was in the security paper as an example.

Additionally I'm considering making the two functions impl'd onto the struct,
so that way they can be called on any capability struct, as I think it would
be nice and ergonomic but not sure what others would think.

This is the spec inside the paper

\begin{verbatim}
CAP := {
    target, accessor : ObjectId,
    permissions, flags : BitField,
    gates: Gates,
    revocation : Revoc,
    siglen: Length,
    sig: u8[],
}
\end{verbatim}


\end{document}
